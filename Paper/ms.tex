%  LaTeX support: latex@mdpi.com
%  In case you need support, please attach any log files that you could have, and specify the details of your LaTeX setup (which operating system and LaTeX version / tools you are using).

%=================================================================

% LaTeX Class File and Rendering Mode (choose one)
% You will need to save the "mdpi.cls" and "mdpi.bst" files into the same folder as this template file.

%=================================================================

\documentclass[journal,article,accept,moreauthors,pdftex,12pt,a4paper]{mdpi}
%--------------------
% Class Options:
%--------------------
% journal
%----------
% Choose between the following MDPI journals:
% actuators, administrativesciences, aerospace, agriculture, agronomy, algorithms, animals, antibiotics, antibodies, antioxidants, appliedsciences, arts, atmosphere, atoms, axioms, batteries, behavioralsciences, beverages, bioengineering, biology, biomedicines, biomimetics, biomolecules, biosensors, brainsciences, buildings, cancers, catalysts, cells, challenges, chemosensors, children, chromatography, climate, coatings, computation, computers, cosmetics, crystals, data, dentistryjournal, diagnostics, diseases, diversity, econometrics, economies, education, electronics, energies, entropy, environments, epigenomes, fermentation, fibers, foods, forests, futureinternet, galaxies, games, gels, genealogy, genes, geosciences, geriatrics, healthcare, horticulturae, humanities, hydrology, informatics, information, inorganics, insects, ijerph, ijfs, ijms, ijns, ijgi, jcdd, jcm, jdb, jfb, jfmk, jimaging, jof, joi, jlpea, jmse, jpm, jrfm, jsan, land, languages, laws, life, lubricants, machines, marinedrugs, materials, mathematics, medicalsciences, membranes, metabolites, metals, microarrays, micromachines, microorganisms, minerals, molbank, molecules, nanomaterials, ncrna, nutrients, pathogens, pharmaceuticals, pharmaceutics, pharmacy, philosophies, photonics, plants, polymers, processes, proteomes, publications, recycling, religions, remotesensing, resources, risks, robotics, safety, sensors, sinusitis, socialsciences, societies, sports, standards, sustainability, symmetry, systems, technologies, toxics, toxins, universe, vaccines, veterinarysciences, viruses, water
%---------
% article
%---------
% The default type of manuscript is article, but could be replaced by using one of the class options:
% article, review, communication, commentary, bookreview, correction, addendum, editorial, changes, supfile, casereport, comment, conceptpaper, conferencereport, meetingreport, discussion, essay, letter, newbookreceived, opinion, projectreport, reply, retraction, shortnote, technicalnote, creative, datadescriptor (for journal Data), briefreport, hypothesis, interestingimage
%----------
% submit
%----------
% The class option "submit" will be changed to "accept" by the Editorial Office when the paper is accepted. This will only make changes to the frontpage (e.g. the logo of the journal will get visible), the headings, and the copyright information. Journal info and pagination for accepted papers will also be assigned by the Editorial Office.
% Please insert a blank line is before and after all equation and eqnarray environments to ensure proper line numbering when option submit is chosen
%------------------
% moreauthors
%------------------
% If there is only one author the class option oneauthor should be used. Otherwise use the class option moreauthors.
%---------
% pdftex
%---------
% The option "pdftex" is for use with pdfLaTeX only. If eps figure are used, use the optioin "dvipdfm", with LaTeX and dvi2pdf only.

%=================================================================
\setcounter{page}{1}
\lastpage{x}
\doinum{10.3390/------}
\pubvolume{xx}
\pubyear{2015}
%\externaleditor{Academic Editor: xx}
\history{Received: xx / Accepted: xx / Published: xx}

\usepackage{dsfont}
\usepackage[utf8]{inputenc}

\newcommand{\xx}{\boldsymbol{x}}
\newcommand{\dx}{d\boldsymbol{x}}
\newcommand{\data}{\boldsymbol{D}}

%------------------------------------------------------------------
% The following line should be uncommented if the LaTeX file is uploaded to arXiv.org
%\pdfoutput=1

%=================================================================

% Add packages and commands to include here
% The amsmath, amsthm, amssymb, hyperref, caption, float and color packages are loaded by the MDPI class.
%\usepackage{graphicx}
%\usepackage{subfigure,psfig}

%=================================================================
%% Please use the following mathematics environments:
% \theoremstyle{mdpi}
% \newcounter{thm}
% \setcounter{thm}{0}
% \newcounter{ex}
% \setcounter{ex}{0}
% \newcounter{re}
% \setcounter{re}{0}
%
% \newtheorem{Theorem}[thm]{Theorem}
% \newtheorem{Lemma}[thm]{Lemma}
% \newtheorem{Corollary}[thm]{Corollary}
% \newtheorem{Proposition}[thm]{Proposition}
%
% \theoremstyle{mdpidefinition}
% \newtheorem{Characterization}[thm]{Characterization}
% \newtheorem{Property}[thm]{Property}
% \newtheorem{Problem}[thm]{Problem}
% \newtheorem{Example}[ex]{Example}
% \newtheorem{ExamplesandDefinitions}[ex]{Examples and Definitions}
% \newtheorem{Remark}[re]{Remark}
% \newtheorem{Definition}[thm]{Definition}
%% For proofs, please use the proof environment (the amsthm package is loaded by the MDPI class).

%=================================================================

% Full title of the paper (Capitalized)
\Title{Nested sampling with multiple scalars}

% Authors (Add full first names)
\Author{Brendon J. Brewer$^{1,}$* and Ewan Cameron$^{2}$}

% Affiliations / Addresses (Add [1] after \address if there is only one affiliation.)
\address{%
$^{1}$ Department of Statistics, The University of Auckland, Private Bag 92019,
Auckland 1142, New Zealand\\
$^{2}$ Spatial Ecology and Epidemiology Group, Tinbergen Building, Department
of Zoology, University of Oxford, South Parks Road, Oxford, UK}

%\contributed{$^\dagger$ These authors contributed equally to this work.}

% Contact information of the corresponding author (Add [2] after \corres if there are more than one corresponding author.)
\corres{{\tt bj.brewer@auckland.ac.nz}}

% Abstract (Do not use inserted blank lines, i.e. \\)
\abstract{The abstract goes here.}

% Keywords: add 3 to 10 keywords
\keyword{nested sampling; bayesian computation; statistical mechanics}

% The fields PACS, MSC, and JEL may be left empty or commented out if not applicable
%\PACS{}
%\MSC{}
%\JEL{}

% If this is an expanded version of a conference paper, please cite it here: enter the full citation of your conference paper, and add $^\dagger$ in the end of the title of this article.
%\conference{}

%%%%%%%%%%%%%%%%%%%%%%%%%%%%%%%%%%%%%%%%%%
% For journal Data:

%\dataset{DOI number or link to the deposited data set in cases where the data set is published or set to be published separately. If the data set is submitted and will be published as a supplement to this paper in the journal Data, this field will be filled by the editors of the journal. In this case, please make sure to submit the data set as a supplement when entering your manuscript into our manuscript editorial system.}
%\datasetlicense{license under which the data set is made available (CC0, CC-BY, CC-BY-SA, CC-BY-NC, etc.)}

%%%%%%%%%%%%%%%%%%%%%%%%%%%%%%%%%%%%%%%%%%

\begin{document}

%%%%%%%%%%%%%%%%%%%%%%%%%%%%%%%%%%%%%%%%%%

\section{Introduction}

Nested Sampling \citep[NS][]{skilling} is an effective and popular
Monte Carlo algorithm for Bayesian computation and
statistical mechanics
\citep{2009arXiv0906.3544P, 2014PhRvE..89b2302P, 2015arXiv150303404B}.
In a Bayesian inference problem with unknown parameters denoted collectively
by a vector $\xx$, the
posterior distribution for the parameters given data $\data$ (and prior
information $I$) is:
\begin{eqnarray}
p(\xx | \data, I) &=&
\frac{p(\xx | I)p(\data | \xx, I)}{p(\data | I)}\\
&=& \frac{\pi(\xx)L(\xx)}{Z}
\end{eqnarray}
where $\pi(\xx)$ is the prior distribution, $L(\xx)$ is the likelihood
function, and $Z$ is the normalising constant, known as the
``marginal likelihood'' or the ``evidence'':
\begin{eqnarray}
Z &=& \int \pi(\xx) L(\xx) \, \dx.\label{eqn:evidence}
\end{eqnarray}
It begins by drawing particles from the
prior $\pi(\xx)$ and successively imposing constraints on the value of
the likelihood $L(\xx)$ that compress the prior mass by a
factor that is approximately known.
This enables the calculation of the marginal likelihood $Z$
and properties of the posterior $\pi(\xx)L(\xx)/Z$.
NS is required because Equation~\ref{eqn:evidence}
is the expected value of $L$ with respect to a very heavy-tailed distribution
(the prior for the $L$ value implied by $\pi(\xx)$). Hence, there is a strong
connection between NS and ideas from rare event simulation \citep{walter}.

NS also allows us to calculate the properties of any other distribution that
is in some sense intermediate
between the prior and the posterior. For example, we might be interested in
a ``power posterior'' where the likelihood is raised to a power $\beta$:
\begin{eqnarray}
p(\xx; \beta) &=& \frac{\pi(\xx)L(\xx)^\beta}{Z(\beta)}\label{eqn:power_posterior}
\end{eqnarray}
The normalisation and posterior samples from this distribution can be obtained
from the original Nested Sampling run. An example application, in the case of
a Bayesian model with a ``gaussian noise'' assumption in the likelihood,
computing $p(\xx; \beta)$ for $\beta \neq 1$ allows us to explore what the
posterior distribution would have been if the noise variance had been greater.
Relative to alternative approaches such as annealing, Nested Sampling continues
to work when the problem contains a {\it phase transition}.

In statistical mechanics, Equation~\ref{eqn:power_posterior} defines the
family of {\it canonical distributions}, usually written as:
\begin{eqnarray}
p(\xx; \beta) &=& \frac{\pi(\xx)\exp[-\beta E(\xx)]}{Z(\beta)}
\end{eqnarray}
where $E(\xx)$ is the energy function. In this context we usually want to
compute $Z(\beta)$ as a function of $\beta$, which is called the
{\it partition function}.

In some inference and
statistical mechanics problems, there are two or more scalar functions of
$\xx$ that are relevant. Suppose our prior is $\pi(\xx)$ as before, and
we obtain testable information that fixes the expected values of two scalar
functions of $\xx$, $S_1(\xx)$ and $S_2(\xx)$. It is well known that the
updated probability distribution that takes into account the constraints is
of the ``canonical'' form:

\begin{eqnarray}
p(\xx; \beta_1, \beta_2) &=& \frac{\pi(\xx)\exp\left[\beta_1S_1(\xx)+\beta_2S_2(\xx)\right]}
{Z(\beta_1, \beta_2)}
\end{eqnarray}

If we were only interested in a single canonical distribution, for example
with $\beta_1 = 0.3$ and $\beta_2 = 0.7$, we could estimate its normalising
constant and by running standard Nested Sampling with ``likelihood''
$L(\xx) = \exp\left[0.3S_1(\xx) + 0.7S_2(\xx)\right]$. However, usually we
are interested in a range of values for $\beta_1$ and $\beta_2$, and we
want to know the normalisation $Z(\beta_1, \beta_2)$, called the
partition function. This work describes
progress towards solving this class of problems while maintaining the benefits
of Nested Sampling, such as the ability to cope with first-order phase
transitions.

\section{Properties of Nested sampling}

We seek an algorithm for computing the partition function
$Z(\beta_1, \beta_2)$ for a range of values of the $\beta$s, from a single
run. To be considered a variant of Nested Sampling, we require the method
to satisfy the following requirements:
\begin{enumerate}
\item The algorithm should begin with $N$ points drawn from the prior $\pi(\xx)$.
\item The algorithm should seek to explore regions where the values of
$S_1(\xx)$ and $S_2(\xx)$ are greater than what would be expected from the
prior.
\item The algorithm should try to move through a sequence of probability
distributions proportional to $\pi$, but restricted to smaller and smaller
domains for which we can approximately measure the enclosed prior mass.
\item The algorithm should be invariant to monotonic transformations of
$S_1$ and $S_2$, i.e. it should only depend on rankings of $S_1$ and $S_2$
values and not the values themselves.
\end{enumerate}

The algorithm works as follows.


\section{Demonstration Example}
The prior is a uniform distribution over the $n$-dimensional unit hypercube:
\begin{eqnarray}
p(\xx) &=& 1
\end{eqnarray}
as long as $x_i \in [0, 1]$ for all $i$. The scalars are:
\begin{eqnarray}
S_1(\xx) &=& -\sum_{i=1}^n \left(x_i - 0.5\right)^2\\
S_2(\xx) &=& -\sum_{i=1}^n \sin^2\left(4\pi x_i\right)
\end{eqnarray}

For this section, we used $n=1000$ dimensions.

\section{Example: Potts model with some embellishments}

TBD

\section{A Diffusive Variant}

Have one walker, or a few (not a large number).

Explore the prior $\pi(\xx)$, and save sampled values of the scalars
(thinned, presumably). As a function of position $(s_1, s_2)$ there is a well
defined value of the cumulative prior mass
\begin{eqnarray}
M(s_1, s_2) &=& \int \pi(\xx)\mathds{1}\left(S_1(\xx) \leq s_1,
S_2(\xx) \leq s_2\right)\, d\xx
\end{eqnarray}
which can be approximated in a Monte Carlo fashion by
\begin{eqnarray}
\hat{M}(s_1, s_2) &=& \frac{1}{N}\sum_{i=1}^N \mathds{1}\left(S_1^i \leq s_1,
S_2^i \leq s_2\right)
\end{eqnarray}


\acknowledgments{Acknowledgements}

It is a pleasure to thank Gábor Csányi (Cambridge), Livia B. Pártay (Cambridge),
and Robert Baldock (Cambridge) for valuable conversations.

%%%%%%%%%%%%%%%%%%%%%%%%%%%%%%%%%%%%%%%%%%

\authorcontributions{Author Contributions}

Required if more than one author. Authorship must include and be strictly limited to researchers who have substantially contributed to the reported work. Please carefully review our criteria regarding the Qualification for Authorship: \web /instructions.

%%%%%%%%%%%%%%%%%%%%%%%%%%%%%%%%%%%%%%%%%%

\conflictofinterests{Conflicts of Interest}
The authors declare no conflict of interest.

%=================================================================
% References: Variant A
%=================================================================
% Back Matter (References and Notes)
%----------------------------------------------------------
% Style and layout of the references
\bibliographystyle{mdpi}
\makeatletter
\renewcommand\@biblabel[1]{#1. }
\makeatother

\begin{thebibliography}{999} % if there are less than 10 entries, enter a one digit number

\bibitem[Baldock et al.(2015)]{2015arXiv150303404B} Baldock, R.~J.~N., 
P{\'a}rtay, L.~v.~B., Bart{\'o}k, A.~P., Payne, M.~C., Cs{\'a}nyi, G.\ 
2015.\ Determining pressure-temperature phase diagrams of materials.\ ArXiv 
e-prints arXiv: 1503.03404. 

\bibitem[P{\'a}rtay et al.(2014)]{2014PhRvE..89b2302P} P{\'a}rtay, L.~B., 
Bart{\'o}k, A.~P., Cs{\'a}nyi, G.\ 2014.\ Nested sampling for materials: 
The case of hard spheres.\ Physical Review E 89, 022302. 

\bibitem[P{\'a}rtay et al.(2009)]{2009arXiv0906.3544P} P{\'a}rtay, L.~B., 
Bart{\'o}k, A.~P., Cs{\'a}nyi, G.\ 2009.\ Efficient sampling of atomic 
configurational spaces.\ ArXiv e-prints arXiv: 0906.3544. 

\bibitem[\protect\citeauthoryear{Skilling}{2006}]{skilling} Skilling, J., 2006, Nested Sampling for General Bayesian Computation, Bayesian Analysis 4, pp. 833-860.

\bibitem[Walter(2015)]{walter}
Walter, C.\ Point Process-based Monte Carlo estimation.\ arXiv: 1412.6368.
%% Reference 1
%\bibitem{ref-journal}
%Lastname, F.; Author, T. The title of the cited article. {\em Journal Abbreviation} {\bf 2008}, {\em 10}, 142-149.

%% Reference 2
%\bibitem{ref-book}
%Lastname, F.F.; Author, T. The title of the cited contribution. In {\em The Book Title}; Editor, F., Meditor, A., Eds.; Publishing House: City, Country, 2007; pp. 32-58.

\end{thebibliography}

%=================================================================
% References:  Variant B
%=================================================================
% Use the following option to include external BibTeX files:
%\bibliography{lite}
%\bibliographystyle{mdpi}

%%%%%%%%%%%%%%%%%%%%%%%%%%%%%%%%%%%%%%%%%%

%\abbreviations{Abbreviations/Nomenclature}
%
%Main text.

%%%%%%%%%%%%%%%%%%%%%%%%%%%%%%%%%%%%%%%%%%

%\appendix
%\section{Appendix Title}
%
%Main text.

\end{document}

